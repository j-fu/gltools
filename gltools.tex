% $Id: gltools.tex,v 2.16 2003/03/28 16:47:48 fuhrmann Exp $
\documentclass[a4paper]{article}
\pagestyle{headings}
\usepackage{html}
\usepackage{epsfig}
\usepackage{verbatim}
\newcommand{\href}[2]{\htmladdnormallinkfoot{#1}{#2}}
\newcommand{\xref}[1]{{\tt #1} (\ref{func:#1})}
\newcommand{\func}[1]{\subsubsection{#1}\label{func:#1}}
\setlength{\parindent}{0pt}
\sloppy
\title{gltools - an OpenGL based on-line graphics toolbox}
\author{J\"urgen Fuhrmann \and Hartmut Langmach}
\date{$Date: 2003/03/28 16:47:48 $ $Revision: 2.16 $}
\begin{document}
\maketitle

{\em gltools}  is a collection of OpenGL rendering utilities.
It consists of three layers:
\begin{description}
\item[glwin:] Interface to the windowing system.
\item[glrnd:] Management of an orthogonal  rendering volume.
\item[glmesh:] Rendering utilities for functions on simplicial meshes
  including 3D isosurfaces and plane sections.
\end{description}
Further, parallel to the system dependent glwin module , 
there is the system independent gleps module
which organizes vector postscript dump.

For Information, see also the 
\htmladdnormallinkfoot{gltools homepage}{http://www.wias-berlin.de/~gltools}.


\section{Introduction}
\subsection{What is the intention of  {\em gltools} ?}


You have program and want to make it some pictures. You are developing
3D code and are unable to find any errors in your data arrays. You
look for a replacement of GKS which no one seems to have anymore.

Many      graphics        packages,      as       the       
\htmladdnormallinkfoot{GLUTtoolkit}{http://www.sgi.com/Technology/openGL/glut.html} 
(which    has nearly      the  same     intentions    as {\em    gltools}),
\htmladdnormallinkfoot{AVS}{http://www.avs.com}   
and  \htmladdnormallinkfoot{GRAPE}{http://www.mathematik.uni-freiburg.de/Grape/grape.html} 
provide a perfect environment.

But for them, one has  to register existing code  as a callback and/or
one has  to translate the existing  data structures  into those of the
graphics package.  This may  be  fairly time consuming and  difficult.
Also, one wants to control the graphics package via the existing code,
not vice versa.  This is where the {\em gltools} package comes in. 

It should enable one to get easy access to the   OpenGL world
{\em on-line}
from one's  {\em own} data structures within {\em existing code}.

It has been tested with OpenGL on SGI Irix 6.x and Compaq Tru64 UNIX 
as well as with the 
\htmladdnormallinkfoot{Mesa package}{http://www.ssec.wisc.edu/~brianp/Mesa.html}
 of Brian Paul. It {\em should} 
compile with any ANSI-C compiler.

\subsection{What is not the intention of {\em gltools} ?}
It is intended to keep this package compact. 
There are two main reasons for this:
\begin{itemize}
\item the limited time of the authors 
\item the ease of use of {\em gltools} for the programmer.
\end{itemize}

So, it  is until now not planned  to incorporate menu control into the
package.  An easy  (and most preferred  by the author)  way to provide
menus would be a callback to a scripting language like {\em tcl} which
could be equipped with a  menu system, but this  makes sense only when
the whole code is embedded into such a language. 

\subsection{Sample code}
\paragraph{glwexample-appctrl.c} contains a simple test
program with a GL window in application control mode.

\paragraph{glwexample-evctrl.c} contains a simple test
program with a GL window in event control mode.

\paragraph{glrexample.c} contains a simple test
program with a GL rendere.

\paragraph*{glview.c} contains a sample program which can be used to render
function data on rectangular meshes. It is used as a test program
for {\em glrnd} and {\em glmesh}.

\subsection{To Do}
\begin{itemize}
\item vector field rendering 
\item handle arbitrary clipping planes -- this is
only a question of a clever  keyboard and  mouse interface, 
{\em glmesh} already does everything.
\item relate shown values in the title to real values.
\item control rotation axes : keys X,Y,Z
\end{itemize}


\input glwin.h-tex
\input glrnd.h-tex
\input glmesh.h-tex
\input gleps.h-tex
\end{document}



% $Log: gltools.tex,v $
% Revision 2.16  2003/03/28 16:47:48  fuhrmann
% pdelib1.15_alpha1
%
% Revision 2.15  2003/03/28 11:20:25  fuhrmann
% pdelib2.0_alpha1
%
% Revision 2.14  1999/12/21 17:18:55  fuhrmann
% doc update for gltools-2-3
%
% Revision 2.13  1998/12/18 14:35:21  fuhrmann
% Distribution stuff
%
% Revision 2.12  1997/12/12 12:45:38  fuhrmann
% new doc mechanism
%
% Revision 2.11  1997/11/27  19:03:09  fuhrmann
% glwRecord stuff, PAL-Format, tex-file for keys
%
% Revision 2.10  1997/11/25  10:12:28  fuhrmann
% slightly changed input statement
%
% Revision 2.9  1997/11/24  17:45:16  fuhrmann
% introduction removed
%
% Revision 2.8  1997/10/27  14:39:46  fuhrmann
% doc stuff
%
% Revision 2.7  1997/05/20  15:59:54  fuhrmann
% *** empty log message ***
%
% Revision 2.6  1997/05/19  18:09:31  fuhrmann
% func,xref
%
% Revision 2.5  1997/05/19  15:46:53  fuhrmann
% .h-tex include files
%
% Revision 2.4  1997/04/04  12:25:52  fuhrmann
% rcs style, include *.tex
%
% Revision 2.3  1996/11/06  20:39:01  fuhrmann
% documentation improved
%
% Revision 2.2  1996/11/06  18:28:34  fuhrmann
% input files as .h-tex
%
% Revision 2.1  1996/09/23  17:05:13  fuhrmann
% switched to xdoc
%
% Revision 2.0  1996/02/15  19:57:10  fuhrmann
% First meta-stable distribution
%
% Revision 1.2  1996/02/15  14:17:45  fuhrmann
% glwin& glrnd besser dokumentiert
%
% Revision 1.1  1995/10/20  15:45:38  fuhrmann
% Initial revision
%
